\chapter{Change Detection}

\section{Simple Detectors}
\label{sec:SimpleDetectors}
\subsection{Mean Difference}
\label{sec:MeanDifference}

The simplest change detector is based on the pixel-wise differencing
of image values: 
\begin{equation}
I_{D}(i,j)=I_{2}(i,j)-I_{1}(i,j).
\end{equation}

In order to make the algorithm robust to noise, one actually uses
local means instead of pixel values.

\input{DiffChDet}

\subsection{Ratio Of Means}
\label{sec:RatioOfMeans}

This detector is similar to the previous one except that it uses a
ratio instead of the difference:
\begin{equation}
\displaystyle I_{R}(i,j) = \frac{\displaystyle I_{2}(i,j)}{\displaystyle I_{1}(i,j)}.
\end{equation}

The use of the ratio makes this detector robust to multiplicative
noise which is a good model for the speckle phenomenon which is
present in radar images.

In order to have a bounded and normalized detector the following
expression is actually used:


\begin{equation}
\displaystyle I_{R}(i,j) = 1 - min \left(\frac{\displaystyle I_{2}(i,j)}{\displaystyle I_{1}(i,j)},\frac{\displaystyle I_{1}(i,j)}{\displaystyle I_{2}(i,j)}\right).
\end{equation}


\input{RatioChDet}


\section{Statistical Detectors}
\label{sec:StatisticalDetectors}

\subsection{Distance between local distributions}
\label{sec:KullbackLeiblerDistance}

This detector is similar to the ratio of means detector (seen in the 
previous section page~\pageref{sec:RatioOfMeans}). Nevertheless, 
instead of the comparison of means, the comparison is performed to
the complete distribution of the two Random Variables (RVs)~\cite{Inglada03}.

The detector is based on the Kullback-Leibler distance between probability 
density functions (pdfs). In the neighborhood of each pixel of the pair 
of images $I_1$ and $I_2$ to be compared, the distance between local pdfs 
$f_1$ and $f_2$ of RVs $X_1$ and $X_2$ is evaluated by:
\begin{align}
  {\cal K}(X_1,X_2) &= K(X_1|X_2) + K(X_2|X_1) \\
  \text{with} \qquad
  K(X_j | X_i) &= \int_{\mathbbm{R}} 
      \log \frac{f_{X_i}(x)}{f_{X_j}(x)} f_{X_i}(x) dx,\qquad i,j=1,2.
\end{align}
In order to reduce the computational time, the local pdfs $f_1$ and $f_2$ 
are not estimated through histogram computations but rather by a cumulant
expansion, namely the Edgeworth expansion, with is based on the 
cumulants of the RVs:
\begin{equation}\label{eqEdgeworthExpansion}
f_X(x) = \left( 1 + \frac{\kappa_{X;3}}{6} H_3(x) 
					+ \frac{\kappa_{X;4}}{24} H_4(x)
					+ \frac{\kappa_{X;5}}{120} H_5(x)
					+ \frac{\kappa_{X;6}+10 \kappa_{X;3}^2}{720} H_6(x) \right) {\cal G}_X(x).
\end{equation}
In eq.~\eqref{eqEdgeworthExpansion}, ${\cal G}_X$ stands for the Gaussian pdf
which has the same mean and variance as the RV $X$. The $\kappa_{X;k}$
coefficients are the cumulants of order $k$, and $H_k(x)$ are the 
Chebyshev-Hermite polynomials of order $k$ (see~\cite{Inglada07} for deeper
explanations).

\input{KullbackLeiblerDistanceChDet}

\subsection{Local Correlation}
\label{sec:LocalCorrelation}
The correlation coefficient measures the likelihood of a linear
relationship between two random variables:
\begin{equation}
\begin{split}
I_\rho(i,j) &= \frac{1}{N}\frac{\sum_{i,j}(I_1(i,j)-m_{I_1})(I_2(i,j)-m_{I_2})}{\sigma_{I_1}
\sigma_{I_2}}\\
& = \sum_{(I_1(i,j),I_2(i,j))}\frac{(I_1(i,j)-m_{I_1})(I_2(i,j)-m_{I_2})}{\sigma_{I_1}
\sigma_{I_2}}p_{ij}
\end{split}
\end{equation}

where $I_1(i,j)$ and $I_2(i,j)$ are the pixel values of the 2 images and
$p_{ij}$ is the joint probability density. This is like using a linear model:
\begin{equation}
I_2(i,j) = (I_1(i,j)-m_{I_1})\frac{\sigma_{I_2}}{\sigma_{I_1}}+m_{I_2}
\end{equation}
for which we evaluate the likelihood with  $p_{ij}$.

With respect to the difference detector, this one will be robust to
illumination changes.
\input{CorrelChDet.tex}

\section{Multi-Scale Detectors}
\label{sec:MultiScaleDetectors}

\subsection{Kullback-Leibler Distance between distributions}
\label{sec:KullbackLeiblerProfile}

This technique is an extension of the distance between distributions 
change detector presented in section~\ref{sec:KullbackLeiblerDistance}.
Since this kind of detector is based on cumulants estimations through
a sliding window, the idea is just to upgrade the estimation of the cumulants
by considering new samples as soon as the sliding window is increasing in size.

Let's consider the following problem: how to update the moments when a
$N+1^{th}$ observation $x_{N+1}$ is added to a set of observations $\{x_1, x_2, \ldots,
x_N\}$ already considered.
The evolution of the central moments may be characterized by:
\begin{align}\label{eqMomentN}
	\mu_{1,[N]} & = \frac{1}{N} s_{1,[N]} \\
	\mu_{r,[N]} & = \frac{1}{N} \sum_{\ell = 0}^r \binom{r}{\ell} 
									\left( -\mu_{1,[N]} \right)^{r-\ell}
									s_{\ell,[N]}, \notag
\end{align}
where the
notation $s_{r,[N]} = \sum_{i=1}^N x_i^r$ has been used.
Then, Edgeworth series is updated also by transforming moments to
cumulants by using:
\begin{equation}\label{eqCumsMoms}
  \begin{split}
\kappa_{X;1} &= \mu_{X;1}\\
\kappa_{X;2} &= \mu_{X;2}-\mu_{X;1}^2\\
\kappa_{X;3} &= \mu_{X;3} - 3\mu_{X;2} \mu_{X;1} + 2\mu_{X;1}^3\\
\kappa_{X;4} &= \mu_{X;4} - 4\mu_{X;3} \mu_{X;1} - 3\mu_{X;2}^2 + 12 \mu_{X;2} \mu_{X;1}^2 - 6\mu_{X;1}^4.
  \end{split}
\end{equation}
It yields a set of images that represent the change measure according to an
increasing size of the analysis window.

\input{KullbackLeiblerProfileChDet}

\section{Multi-components detectors}

\subsection{Multivariate Alteration Detector}

\input{MultivariateAlterationDetector}



