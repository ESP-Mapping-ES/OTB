\chapter{Reading and Writing Auxiliary Data}
\index{Auxiliary data}
\label{sec:ReadingAuxData}

As we have seen in the previous chapter, OTB has a great capability to
read and process images. However, images are not the only type of data
we will need to manipulate. Images are characterized by a regular
sampling grid. For some data, such as Digital Elevation Models (DEM)
or Lidar, this is too restrictive and we need other representations.

Vector data are also used to represent cartographic objects,
segmentation results, etc: basically, everything which can be seen as
points, lines or polygons. OTB provides functionnalities for accessing
this kind of data.

\section{Reading DEM Files}
\index{Digital elevation model}
\label{sec:ReadDEM}
\input{DEMToImageGenerator.tex}

\section{Elevation management with OTB}
\ifitkFullVersion
\label{sec:DEMHandler}
\fi
\input{DEMHandlerExample.tex}

More examples about representing DEM are presented in section~\ref{sec:ViewingAltitudeImages}.

\section{Reading and Writing Shapefiles and KML}
\index{Auxiliary data!vector data}
\index{Auxiliary data!shapefile}
\index{Auxiliary data!KML}
\label{sec:ReadVectorData}
\input{VectorDataIOExample.tex}

