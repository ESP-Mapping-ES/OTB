\chapter{Orthorectification and Map Projection}\label{sec:Ortho}


\subsection{Evaluating Sensor Model}
\label{sec:EvaluatingSensorModels}

If no appropriate sensor model is available in the image meta-data,
OTB offers the possibility to estimate a sensor model from the image.

\input{EstimateRPCSensorModelExample.tex}


\subsection{Limits of the Approach}
\label{LimitsoftheApproach}

As you may understand by now, accurate geo-referencing needs accurate
DEM and also accurate sensor models and parameters. In the case where
we have several images acquired over the same area by different
sensors or different geometric configurations, geo-referencing (geographical coordinates) or ortho-rectification
(cartographic coordinates) is not usually enough. Indeed, when working
with image series we usually want to compare them (fusion, change
detection, etc.) at the pixel level.\\

Since common DEM and sensor parameters do not allow for such an
accuracy, we have to use clever strategies to improve the
co-registration of the images. The classical one consists in refining
the sensor parameters by taking homologous points between the images
to co-register. This is called bundle block adjustment and will be
implemented in coming versions of OTB.

Even if the model parameters are refined, errors due to DEM accuracy
can not be eliminated. In this case, image to image registration can
be applied. These approaches are presented in chapters
\ref{chap:ImageRegistration} and \ref{sec:DisparityMapEstimation}.

\section{Orthorectification with OTB}
\ifitkFullVersion
\label{sec:OrthorectificationwithOTB}
\fi
\input{OrthoRectificationExample.tex}

\section{Vector data projection manipulation}
\ifitkFullVersion
\label{sec:VectorDataProjection}
\fi
\input{VectorDataProjectionExample.tex}

\section{Geometries projection manipulation}
\ifitkFullVersion
\label{sec:GeometriesProjection}
\fi
\input{GeometriesProjectionExample.tex}

\section{Vector data area extraction}
\ifitkFullVersion
\label{sec:VectorDataAreaExtraction}
\fi
\input{VectorDataExtractROIExample.tex}
